\documentclass[11pt,ngerman,a4paper]{article}
%Gummi|061|=)
\usepackage{amsmath}
\usepackage{a4wide}
\usepackage{url}
\usepackage{amsthm}
\usepackage{amsbsy}
\usepackage{amssymb}
\usepackage[utf8]{inputenc}
\usepackage{rotating} 
\usepackage{here}
\usepackage{graphicx}
\usepackage{paralist}
\usepackage{selinput}
\usepackage[separate-uncertainty=true]{siunitx}
\usepackage{booktabs}
\sisetup{}
\SelectInputMappings{%
adieresis={ä},
germandbls={ß},
}
\title{\textbf{Versuch V604: Beugung am Spalt}}
\author{Martin Bieker\\
		Julian Surmann\\
		\\
		Durchgef\"{u}hrt am 10.06.2014\\
		TU Dortmund}
\date{}
\usepackage{graphicx}
\begin{document}
\renewcommand\tablename{Tabelle}
\renewcommand\figurename{Abbildung}
\maketitle
\thispagestyle{empty}
\newpage
\clearpage
\setcounter{page}{1}


\section{Einleitung}
Im folgenden Versuch wird die Beugung von Licht an Einzelspalten und einem Doppelspalt untersucht. Unter Beugung versteht man die Abweichung der Ausbreitung von Licht hinter kleinen Öffnungen von den Gesetzen der geometrischen Optik.
\section{Theorie}

\section{Aufbau}

\section{Auswertung}
\subsection{Mikroskopische Vermessung der Spalte}
Das in diesem Versuch zur Vermessung der Beugungsobjekte verwendete Mikroskop besitzt willkürlich unterteilte Skala. Daher muss dieses zunächst mit Hilfe eines Objektmikrometers geeicht werden. Als Umrechnungsfaktor $f$ ergibt sich:
\[
f = \num{3.53e-05}\,\frac{\si{\meter}}{\mathrm{Einheit}}
\]
Die gemessenen und umgerechneten Werte für die Spaltbreite $b$ und den Abstand der Spalte beim Doppelspalt $g$ befinden sich in Tabelle 1.
\begin{table}[h]
\centering
\begin{tabular}{lSSSS}

\toprule
&$\frac{b}{\mathrm{Einheit}}$ & $\frac{b}{\si{\meter}}$ & $\frac{g}{\mathrm{Einheit}}$ &$\frac{g}{\si{\meter}}$\\
\midrule
Einzelspalt A & 1.9  &6.71e-5 & &\\
Einzelspalt B & 3.8  &1.34e-4 &&\\
Doppelspalt   & 4.6  & 1.63e-4&20.8&7.35e-4\\
\bottomrule
\end{tabular}
\end{table}


\subsection{Beungungsmuster am Einzelspalt}
\subsection{Beugungsmuster am Doppelspalt}
\section{Diskussion}

\section{Quellen}
\begin{enumerate}[{[}1{]}]
\item Entnommen der Praktikumsanleitung \textit{} der TU Dortmund. \\
Download am 16.06.14 unter:\\
 \url{http://129.217.224.2/HOMEPAGE/PHYSIKER/BACHELOR/AP/SKRIPT/V406.pdf}
\end{enumerate}
\section{Anhang}
\begin{itemize}
\item Auszug aus dem Messheft
\end{itemize}
\end{document}
