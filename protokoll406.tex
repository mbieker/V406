\documentclass[11pt,ngerman,a4paper]{article}
%Gummi|061|=)
\usepackage{amsmath}
\usepackage{a4wide}
\usepackage{url}
\usepackage{amsthm}
\usepackage{amsbsy}
\usepackage{amssymb}
\usepackage[utf8]{inputenc}
\usepackage{rotating} 
\usepackage{here}
\usepackage{graphicx}
\usepackage{paralist}
\usepackage{selinput}
\usepackage[separate-uncertainty=true]{siunitx}
\usepackage{booktabs}
\sisetup{}
\SelectInputMappings{%
adieresis={ä},
germandbls={ß},
}
\title{\textbf{Versuch V604: Beugung am Spalt}}
\author{Martin Bieker\\
		Julian Surmann\\
		\\
		Durchgef\"{u}hrt am 10.06.2014\\
		TU Dortmund}
\date{}
\usepackage{graphicx}
\begin{document}
\renewcommand\tablename{Tabelle}
\renewcommand\figurename{Abbildung}
\maketitle
\thispagestyle{empty}
\newpage
\clearpage
\setcounter{page}{1}


\section{Einleitung}
Im folgenden Versuch wird die Beugung von Licht an Einzelspalten und einem Doppelspalt untersucht. Unter Beugung versteht man die Abweichung der Ausbreitung von Licht hinter kleinen Öffnungen von den Gesetzen der geometrischen Optik.
\section{Theorie}
Im Fall der Beugung muss Licht als Welle betrachtet werden. Beugungphänomene können dann z.B. mit dem Huygensschen Prinzip und dem Interferenzprinzip erklärt werden. 
\subsection{Beugung am Einfach-Spalt}
Es sind zwei Formen der Beugung zu unterscheiden: Die Fresnel-Beugung und die Fraunhofer-Beugung (Abbildung \ref{abb1}).\newline
Bei der Fresnel-Beugung befinden sich die Lichtquelle und der Beobachtungspunkt (Schirm) im Endlichen. Dabei interferieren im Beobachtungspunkt Strahlen, die unter verschiedenen Winkeln gebeugt wurden. Bei der Fraunhofer-Beugung liegt die Lichtquelle im Unendlichen, d.h. es werden parallele Strahlenbündel emittiert. Mathematisch ist der zweite Fall viel einfacher zu behandeln, auf die Fresnel-Beugung soll hier nicht weiter eingegangen.
Der Länge des verwendete Spaltes beträgt ein Vielfaches der Breite, sodass nur ein eindimensionales Beugungsbild entsteht.
Es fällt eine ebene Welle mit der Feldstärke
\begin{equation}
A(z,t) = A_0 e^{i(wt- \frac{2\pi z}{ \lambda} )}
\end{equation}
pro Längeneinheit der Wellenfront aus der Z-Richtung ein.
Um das Prinzip der Fraunhofer-Beugung zu verwenden, ist der Abstand Beugungsebene-Beobachtungsebene sehr viel größer als die Spaltbreite b.\newline
Um die Amplitude des in Richtung $\phi$ gebeugten Lichtes zu berechnen, muss man über alle Strahlenbündel summieren, die von allen Orten der Spaltöffnung in Richtung $\phi$ emittiert werden. Zwei beliebige Strahlenbündel, die im Spalt eine Strecke x voneinander entfernt sind, haben aufgrund der Wegdifferenz zur Beobachtungsebene eine Phasendifferenz $\delta$ mit
\begin{equation}
\delta = \frac{2 \pi s}{\lambda} = \frac{2 \pi x \sin \phi}{\lambda}
\end{equation}
Aufgrund der infinitesimal kleinen Breiten der Strahlenbündel geht in der Berechnung der Amplitude die Summe in ein Integral über:
\begin{equation}
B (z, t, \phi) = A_0 \int_0^b {e^{i\left(wt-\frac{2 \pi z}{\lambda}+ \delta \right) } dx} = A_0 e^{i \left( wt- \frac{2 \pi z}{\lambda} \right)} \int_0^b {e^{\left( \frac{2 \pi i x \sin \phi}{\lambda} \right) }} dx.
\end{equation}
Durch Integration und Ausklammern ergibt sich über die Eulersche Formel
\begin{equation}
\sin \alpha = \frac{\left( e^{i \alpha} - e^{-i \alpha} \right)}{2i}
\end{equation}
der folgende Ausdruck:
\begin{equation}
B (z,t,\phi) = A_0 e^{i\left( wt- \frac{2 \pi z}{\lambda}\right)} e^{\left( \frac{\pi i b \sin \phi}{\lambda}\right)} \frac{\lambda}{\pi \sin \phi} \sin \left( \frac{\pi b \sin \phi}{\lambda} \right)
\end{equation}
\subsection{Beugung am Doppelspalt}
\subsection{Fraunhofersche Beugung und Fourier-Transformation}
\section{Aufbau}
Eine Skizze des Versuchsaufbaus ist in Abbildung (\ref{abb2}) gezeigt. Auf einer Schiene mit Längenskala sind ein He-Ne-Laser, eine Halterung für den Spalt und die verschiebbare Photodiode befestigt. Der Laser leuchtet auf den Spalt, der zwischen dem Laser und der Photodiode befestigt ist. Die Photodiode kann mit einer Auflösung von $\SI{1/100}{\milli \meter}$ über eine Strecke von \SI{5}{\centi \meter} verschoben werden. Die Photodiode ist an einem empfindlichen Amperemeter angeschlossen.
\section{Durchführung}
\subsection{Messungen mit dem Einfach-Spalt}
Bei dem ersten Einzelspalt (DatenDatenDaten) soll dessen Beugungsfigur punktweise ausgemessen werden. Die Schrittweite beträgt dabei $\SI{1}{\milli \meter}$. 
\section{Auswertung}
\subsection{Mikroskopische Vermessung der Spalte}
Das in diesem Versuch zur Vermessung der Beugungsobjekte verwendete Mikroskop besitzt willkürlich unterteilte Skala. Daher muss dieses zunächst mit Hilfe eines Objektmikrometers geeicht werden. Als Umrechnungsfaktor $f$ ergibt sich:
\[
f = \num{3.53e-05}\,\frac{\si{\meter}}{\mathrm{Einheit}}
\]
Die gemessenen und umgerechneten Werte für die Spaltbreite $b$ und den Abstand der Spalte beim Doppelspalt $g$ befinden sich in Tabelle 1.
\begin{table}[h]
\centering
\begin{tabular}{lSSSS}

\toprule
&$\frac{b}{\mathrm{Einheit}}$ & $\frac{b}{\si{\meter}}$ & $\frac{g}{\mathrm{Einheit}}$ &$\frac{g}{\si{\meter}}$\\
\midrule
Einzelspalt A & 1.9  &6.71e-5 & &\\
Einzelspalt B & 3.8  &1.34e-4 &&\\
Doppelspalt   & 4.6  & 1.63e-4&20.8&7.35e-4\\
\bottomrule
\end{tabular}
\end{table}


\subsection{Beungungsmuster am Einzelspalt}
\subsection{Beugungsmuster am Doppelspalt}
\section{Diskussion}

\section{Quellen}
\begin{enumerate}[{[}1{]}]
\item Entnommen der Praktikumsanleitung \textit{} der TU Dortmund. \\
Download am 16.06.14 unter:\\
 \url{http://129.217.224.2/HOMEPAGE/PHYSIKER/BACHELOR/AP/SKRIPT/V406.pdf}
\end{enumerate}
\section{Anhang}
\begin{itemize}
\item Auszug aus dem Messheft
\end{itemize}
\end{document}
